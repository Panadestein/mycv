\documentclass[11pt,a4paper,sans]{moderncv}
\moderncvstyle{casual}
\moderncvcolor{blue}

%----------------------------------------------------------------------------------------
%	Packages
%----------------------------------------------------------------------------------------
\usepackage[scale=0.82]{geometry} % Reduce document margins
\def\changemargin#1#2{\list{}{\rightmargin#2\leftmargin#1}\item[]}
\let\endchangemargin=\endlist 
\usepackage{fancyhdr}
\usepackage{ragged2e}
\usepackage{mhchem}
\usepackage{natbib}
\renewcommand{\bibsection}{}
\usepackage{xcolor}
\definecolor{unazul}{HTML}{539bd7}
\usepackage{comment}

% Additional macros
\newcommand*{\scholarsocialsymbol}{\includegraphics[height=.7\baselineskip]{Figures/gscholar}}
\newcommand*{\rgatesocialsymbol}{\includegraphics[height=.75\baselineskip]{Figures/rgate}}
\newcommand*{\stackovsocialsymbol}{\includegraphics[height=.75\baselineskip]{Figures/stacko}}

\makeatletter
\renewcommand*{\makeletterclosing}{
  \@closing\\[3em]%
  \includegraphics[scale=.8]{Figures/signature_panades.png}\\% Insert signature
  {Ramón L. Panad\'es-Barrueta}%
  \ifthenelse{\isundefined{\@enclosure}}{}{%
    \\%
    \vfill%
    {\color{color2}\itshape\enclname: \@enclosure}}}
\makeatother

%----------------------------------------------------------------------------------------
%	Contact information
%----------------------------------------------------------------------------------------

\firstname{\LARGE Ramón L.}
\familyname{\Huge Panadés Barrueta}
\title{\href{https://panadestein.github.io}{panadestein.github.io}}
\social[github][github.com/Panadestein]{GitHub} 
\social[linkedin][www.linkedin.com/in/rpanades]{LinkedIn}
\collectionadd[rgate]{socials}{\href{https://www.researchgate.net/profile/Ramon_Panades-Barrueta}{ ResearchGate}}
\collectionadd[scholar]{socials}{\href{https://scholar.google.com/citations?user=IMWuq6wAAAAJ&hl=en}{ Google Scholar}}
\collectionadd[stackov]{socials}{\href{https://stackoverflow.com/users/5661001/panadestein?tab=profile}{ Stackoverflow}}
\homepage{panadestein.github.io} 
\photo[75pt][0.3pt]{Figures/professional_rpanades.jpg}

%----------------------------------------------------------------------------------------
% CV content
%----------------------------------------------------------------------------------------

\begin{document}

\makecvtitle

\vspace{7.0cm}
\section{Professional experience}

\cventry{Present\\01.08.2021}{Postdoctoral fellow}{\href{https://tu-dresden.de/}{\textcolor{unazul}{TU Dresden}}, Large-Scale Theoretical Spectroscopy group  (\href{https://golzegroup.org/}{\textcolor{unazul}{LSTS}})}{Germany}{}{}
\cvitem{}{Project:  Accurate, Exascale-Ready Methods for Theoretical Core-Level Spectroscopy of Complex Materials (\href{https://gepris.dfg.de/gepris/projekt/453275048?language=en}{\textcolor{unazul}{CoreXL}}). Dr. Dorothea Golze's Emmy-Noether group.}
\cvitem{}{This project aims at developing High-Performance theoretical methods for core-level spectroscopy, that will greatly benefit from the new generation of exascale supercomputers. The focus of my research in the group is the development of low-scaling GW-based algorithms for the prediction of XPS and XAS spectra. The former are implemented in the FHI-aims software package. Additional stand-alone libraries will be also developed in the framework of the \href{https://www.nomad-coe.eu/}{\textcolor{unazul}{NOMAD}} CoE.}

\cventry{31.07.2021\\01.11.2020}{Postdoctoral fellow}{\href{https://www.utwente.nl/en}{\textcolor{unazul}{University of Twente}}, Computational Chemical Physics Group  (\href{https://www.utwente.nl/en/tnw/ccp/}{\textcolor{unazul}{CCP}})}{Netherlands}{}{}
\cvitem{}{Project: Targeting Real chemical accuracy at the EXascale (\href{https://trex-coe.eu/}{\textcolor{unazul}{TREX}}). European HPC Centre of Excellence (\href{https://www.hpccoe.eu/index.php/trex-coe/}{\textcolor{unazul}{CoE}})}
\cvitem{}{This European CoE aims at developing open-source high-performance quantum chemistry software tailored to the emerging exascale architectures. As a member of Prof. Claudia Filippi's group, I contributed to the Cornell-Holland Ab-initio Materials Package (CHAMP). During my time in the project, I have employed innovative programming techniques like the Implicit Reference to Parameters method (IRP).}

\clearpage
\section{Education}

\cventry{23.10.2020 \\ 01.10.2017}{PhD in Physics}{\href{www.univ-lille.fr}{\textcolor{unazul}{University of Lille}}, Laboratoire de Physique des Lasers, Atomes et Molécules (\href{phlam.univ-lille.fr}{\textcolor{unazul}{PhLAM}})}{France}{}{}
\cvitem{}{Thesis title: Full quantum simulations of the interaction between atmospheric molecules and model soot particles (available at \href{https://theses.fr/2020LILUR022}{\textcolor{unazul}{theses.fr}})}
\cvitem{}{Supervisor: Prof. Dr. Daniel Pel\'aez-Ruiz (\href{http://www.ismo.u-psud.fr/spip.php?rubrique28&lang=fr}{\textcolor{unazul}{ISMO}}, Université Paris-Saclay)}
\cvitem{}{The subject of my thesis was the development and application of optimization and tensor decomposition algorithms for representing Potential Energy Surfaces. The later were employed in Nuclear Quantum Dynamics calculations with the Multiconfiguration Time-dependent Hartree (MCTDH) method.}
\cventry{23.06.2017\\12.09.2016}{MSc in Physics (International Master 2 Atmospheric Environments)}{\href{www.univ-lille.fr}{\textcolor{unazul}{University of Lille}}, Laboratoire de Physique des Lasers, Atomes et Molécules (\href{phlam.univ-lille.fr}{\textcolor{unazul}{PhLAM}})}{France}{}{}
\cvitem{}{Thesis title: Towards a quantum dynamical description of the photodissociation of $Cl_{2}$ molecule adsorbed on ice}
\cvitem{}{Supervisors: Prof. Dr. Daniel Pel\'aez-Ruiz and Prof. Dr. Maurice Monnerville}
\cventry{01.07.2016\\01.09.2011}{BSc in Radiochemistry}{\href{www.uh.cu}{\textcolor{unazul}{University of Havana}}, Higher Institute of Technologies and Applied Sciences (\href{www.instec.cu}{\textcolor{unazul}{InSTEC}})}{Havana, Cuba}{}{}
\cvitem{}{Thesis title: Mean Potential Phase Space Theory study of the
  $Si({}^{3}P) + OH(X^{2}\Pi) \rightarrow SiO(X^{1}\Sigma^{+}) + H({}^{2}S)$ reaction}
\cvitem{}{Supervisor: Assoc. Prof. Dr. Alejandro Rivero-Santamar\'ia (\href{phlam.univ-lille.fr}{\textcolor{unazul}{PhLAM}}, Université de Lille)}

\section{Publications}

\cvitem{2023}{Panadés-Barrueta, R. L., Nadoveza, N., Gatti, F., and Peláez D.  (2023) \textbf{On the sum-of-products to product-of-sums transformation between analytical low-rank approximations in finite basis representation}, \href{https://doi.org/10.1140/epjs/s11734-023-00928-z}{\textcolor{unazul}{The European Physical Journal Special Topics, 1--8}}}
\cvitem{2023}{Panadés-Barrueta, R. L. and Golze, D. (2023) \textbf{Accelerating core-level \(GW\) calculations by combining the contour deformation approach with the analytic continuation of \(W\)}, \href{https://doi.org/10.1021/acs.jctc.3c00555}{\textcolor{unazul}{Journal of Chemical Theory and Computation 19 (16), 5450–5464}} (\href{https://arxiv.org/abs/2305.15955}{\textcolor{unazul}{arXiv:2305.15955}})}
\cvitem{2023}{Nadoveza, N., Panadés-Barrueta, R. L., Shi, L., Gatti, F., and Peláez D. (2023) \textbf{Analytical high-dimensional operators in canonical polyadic finite basis representation (CP-FBR)}, \href{https://doi.org/10.1063/5.0139224}{\textcolor{unazul}{The Journal of Chemical Physics, 158, 114109}}}
\cvitem{2022}{Shepard, S., Panadés-Barrueta, R. L., Moroni, S., Scemama A., and Filippi, C. (2022) \textbf{Double excitation energies from quantum Monte Carlo using state-specific energy optimization}, \href{https://doi.org/10.1021/acs.jctc.2c00769}{\textcolor{unazul}{Journal of Chemical Theory and Computation, 18, 11, 6722–6731}} (\href{https://arxiv.org/abs/2207.12160}{\textcolor{unazul}{arXiv:2207.12160}})}
\cvitem{2020}{Panadés-Barrueta, R. L. and Peláez, D. (2020). \textbf{Low-Rank Sum-of-Products Finite-Basis-Representation (SOP-FBR) of Potential Energy Surfaces}, \href{https://doi.org/10.1063/5.0027143}{\textcolor{unazul}{The Journal of Chemical Physics, \textbf{153}, 234110.}}}
\cvitem{2019}{Panadés-Barrueta, R. L., Martínez-Núñez, E., \& Peláez, D. (2019). \textbf{Specific Reaction Parameter Multigrid POTFIT (SRP-MGPF): Automatic Generation of Sum-of-Products Form Potential Energy Surfaces for Quantum Dynamical Calculations}, \href{https://doi.org/10.3389/fchem.2019.00576}{\textcolor{unazul}{Frontiers in Chemistry, \textbf{7}, 576.}} Included in the book \href{https://www.frontiersin.org/research-topics/9088/application-of-optimization-algorithms-in-chemistry}{\textcolor{unazul}{Application of Optimization Algorithms in Chemistry}}}
\cvitem{2016}{Panadés-Barrueta, R. L., Rubayo-Soneira, J., Monnerville, M., Larregaray, P., Dayou, F., and Rivero-Santamar\'ia, A. (2016). \textbf{Mean Potential Phase Space Theory study of the $\mathbf{Si({}^{3}P) + OH(X^{2}\Pi) \rightarrow SiO(X^{1}\Sigma^{+}) + H({}^{2}S)}$ reaction}, \href{http://www.revistacubanadefisica.org/index.php/rcf/article/view/65}{\textcolor{unazul}{Revista Cubana de Física, \textbf{33(2)}, 102-117.}}}

\section{Honors and Awards}

\cvitem{2022}{\textbf{Poster Commendation Prize} Psi-k Conference
  (\href{https://www.psik2022.net/talksposters-guidelines\#h.s39zr9eq17f7}{\textcolor{unazul}{Psi-k 22}}).
  Lausanne, Switzerland.}
\cvitem{2019}{\textbf{Best Poster Prize} 10\textsuperscript{th} International Meeting on Atomic and Molecular Physics and Chemistry (IMAMPC). Madrid, Spain.}
\cvitem{2018}{\textbf{Best Poster Prize} 6\textsuperscript{th} High Dimensional Quantum Dynamics Workshop (HDQD). Lille, France.}
\cvitem{2018}{\textbf{PCCP Best Poster Prize} 9\textsuperscript{th} International Meeting on Atomic and Molecular Physics and Chemistry (\href{https://phlam.univ-lille.fr/detail-actu/ramon-panades-barrueta-le-prix-poster-du-journal-pccp-du-congres-imampc/}{\textcolor{unazul}{IMAMPC}}). Berlin, Germany.}
\cvitem{2015}{\textbf{ICPC Contestant} Caribbean Finals of the International Collegiate Programming Contest (ACM-ICPC). Havana, Cuba}
\cvitem{2010}{\textbf{IChO Contestant} Highest mark in the National Chemistry Olympiad, and captain of the Cuba Team in the 42\textsuperscript{nd} International Chemistry Olympiad (\href{https://www.icho2010.org/en/results.html}{\textcolor{unazul}{IChO 2010}}). Tokyo, Japan}

\section{Competitive research grants and fellowships}

\cventry{Feb. 2023}{Computer time grant of 12 Mi CPU hours}{}{}{}{Awarded by: Paderborn Center for Parallel Computing (\href{https://pc2.uni-paderborn.de/}{\textcolor{unazul}{PC2}}) \\ Project: Accurate, Exascale-Ready Methods for Theoretical Core-Level Spectroscopy of Complex Materials. }
\cventry{Dec. 2022}{Travel Award of the Graduate Academy}{}{}{}{Awarded by: TU Dresden \href{https://tu-dresden.de/ga}{\textcolor{unazul}{Graduiertenakademie}} \\ Comment: Financed participation on the CECAM workshop \href{https://www.cecam.org/workshop-details/1138}{\textcolor{unazul}{Green's function methods: the next generation 5}} }
\cventry{Oct.-Dec. 2019}{Research Grant German Academic Exchange Service (DAAD)}{}{}{}{Awarded by: Deutscher Akademischer Austausch Dienst \\ Place: Universität Heidelberg, \href{https://www.pci.uni-heidelberg.de/cms/index.html}{\textcolor{unazul}{Theoretical Chemistry Group}} \\ Supervisor: Prof. Dr. Oriol VENDRELL}
\cventry{2016--2017}{Labex CaPPA Fellowship}{}{}{}{Awarded by: Laboratoire d'excellence \href{http://www.labex-cappa.fr/master-atmospheric-environment/fellowships}{\textcolor{unazul}{CaPPA}} \\ Place: University of Lille }

\section{Seminars and conferences}

\cventry{March 2023}{Accelerating core-level GW calculations using the CD approach with analytic continuation of W}{Contributed talk}{}{}{\href{https://www.dpg-verhandlungen.de/year/2023/conference/skm/part/o/session/35}{\textcolor{unazul}{DPG-Frühjahrstagung der Sektion Kondensierte Materie}} \\ Dresden, Germany}
\cventry{March 2023}{Accelerating core-level GW calculations by combining the contour deformation with the analytic continuation of W}{Poster}{}{}{\href{https://www.lorentzcenter.nl/accelerating-theoretical-spectroscopy-for-complex-multiscale-materials.html}{\textcolor{unazul}{Accelerating theoretical spectroscopy for complex multiscale materials}} \\ Leiden, Netherlands}
\cventry{November 2022}{Accelerating core-level GW calculations by combining the contour deformation with the analytic continuation of W}{Poster}{}{}{\href{https://www.cecam.org/workshop-details/1138}{\textcolor{unazul}{Green's function methods: the next generation 5}} \\ Toulouse, France}
\cventry{August 2022}{Accelerating core-level GW calculations by combining the contour deformation with the analytic continuation of W}{Poster}{}{}{\href{https://panadestein.github.io/poster_psik/index.html}{\textcolor{unazul}{Psi-k 2022 Conference}} \\ Lausanne, Switzerland}
\cventry{December 2021}{An overview of GW and its applications to core level spectroscopy}{Seminar talk}{}{}{\href{https://panadestein.github.io/gw_talk/sem_rpanades.html}{\textcolor{unazul}{Group seminar}} \\ TU Dresden, Germany}
\cventry{November 2020}{A \textit{gentle} introduction to MCTDH}{Seminar talk}{}{}{\href{https://nbviewer.org/github/Panadestein/mctdh_talk/blob/main/mctdh_intro.pdf}{\textcolor{unazul}{Group seminar}} \\ University of Twente, Netherlands}
\cventry{August 2020}{On the automatic computation of global (intermolecular) potential energy surfaces for quantum dynamical simulations}{Invited Speaker}{}{}{\href{http://spig2020.ipb.ac.rs/invited.html}{\textcolor{unazul}{Symposium and Summer School on Physics of Ionized Gases}} \\ Šabac, Serbia}
\cventry{February 2020}{Automatic computation of global (intermolecular) potential energy surfaces for
  (non) covalently bound systems}{Contributed Talk}{}{}{\href{https://jctmsifn.sciencesconf.org/program/details}{\textcolor{unazul}{Journée Chimie Théorique et Simulation Moléculaire IdF/Nord}} \\ Chimie ParisTech. Paris, France}
\cventry{April 2019}{Automatic computation of potential energy surfaces for covalently bound systems}{Seminar talk}{}{}{Laboratoire de Chimie Physique - Matière et Rayonnement (LCPMR). Paris, France}

\section{Hackathons}

\cventry{August 2023}{\href{https://www.lumi-supercomputer.eu/events/lumi-gpu-hackathon/}{\textcolor{unazul}{LUMI GPU Hackathon}}}{Porting the RI-V integration technique in FHI-aims to HIP (AMD)}{}{CSC – IT Center for Science, Finland (3$^{th}$ supercomputer in TOP500)}{Role: Team leader}
\cventry{May 2023}{\href{https://www.openhackathons.org/s/siteevent/a0C5e000007ZQ6lEAG/se000171}{\textcolor{unazul}{Helmholtz GPU Hackathon}}}{Porting the RI-V integration technique in FHI-aims to CUDA (NVIDIA)}{}{Forschungszentrum Jülich, Germany (13$^{th}$ supercomputer in TOP500)}{Role: Team leader}

\section{Teaching experience}

\cventry{2022}{Modern Topics in Theoretical and Computational Chemistry}{\href{https://golzegroup.org/htdocs/wp-content/uploads/exercises_m19/}{\textcolor{unazul}{course website}}}{}{}{\textbf{Place:} TU Dresden, Germany \space\space\space   \textbf{No. hours:} 4 \space\space\space  \textbf{Language:} English}
\cventry{2021}{Qualification for a position as an assistant professor in a French University}{}{}{}{\textbf{Sections:} 30 and 31 of the \href{https://www.conseil-national-des-universites.fr/cnu/}{\textcolor{unazul}{CNU}} \space\space\space  \textbf{Qualification Numbers:} 21230359242 and 21231359242}
\cventry{2018--2019}{Laboratory of Thermodynamics}{}{}{}{\textbf{Place:} IUT A de Lille (University of Lille), France \space\space\space   \textbf{No. hours:} 64 \space\space\space  \textbf{Language:} French}
\cventry{2013--2016}{Mathematical Analysis and Linear Algebra}{}{}{}{\textbf{Place:} InSTEC (University of Havana), Cuba \space\space\space  \textbf{No. hours:} 72 \space\space\space  \textbf{Language:} Spanish}

\section{Supervised students}

\cventry{2022--2023}{Johannes Günzl}{Master thesis}{}{Arbeitsgruppe für Theoretische Chemie}{TU Dresden}

\section{Reviewing activities}

\cvitem{}{List of Journals I have contributed as a reviewer.}
\cventry{2023}{Physical Review A}{}{(\href{https://journals.aps.org/pra/}{\textcolor{unazul}{PRA}})}{}{}
\cventry{2022}{Physical Review B}{}{(\href{https://journals.aps.org/prb/}{\textcolor{unazul}{PRB}})}{}{}

\section{Research training and attended workshops}

\cventry{June 2023}{5th NOMAD Project Meeting}{Aalto University. Espoo, Finland}{}{}{}
\cventry{June 2022}{NOMAD WP2 Hackathon}{Academic Centre of the University of Latvia. Riga, Latvia}{}{}{}
\cventry{July 2021}{Stochastic Methods in Electronic Structure Theory}{Telluride Science Research Center (\href{https://meetings.telluridescience.org/meetings/workshop-details?wid=874}{\textcolor{unazul}{Virtual Workshop}})}{}{}{}
\cventry{January 2020}{First general assembly of the GDR NBODY: N-body quantum problem in chemistry and physics}{Universit\'e de Lille, France}{}{}{}
\cventry{August 2019}{School EMIE-UP. Multiscale Dynamics in Molecular Systems}{École de Physique des Houches. Haute-Savoie, France}{}{}{}
\cventry{June 2019}{3\textsuperscript{rd} Mini-school on mathematics for theoretical chemistry and physics}{Sorbonne Université, Pierre et Marie Curie. Paris, France}{}{}{}
\cventry{June 2018}{Bridging experiment and theory in precision spectroscopy (BETS) 4\textsuperscript{th} MOLIM Training School}{Nicolaus Copernicus University. Torun, Poland}{}{}{}
\cventry{January 2018}{Label of Theoretical Chemistry \^Ile de France-Nord}{Sorbonne Universit\'e, Pierre et Marie Curie. Paris, France}{}{}{}
\cventry{October 2017}{Quantum Dynamics with the Multi-Configuration Time-Dependent Hartree (MCTDH) method: future and perspectives}{Universit\'e Paris-Saclay. Paris, France}{}{}{}
\cventry{March 2017}{International Conference on Aerosol Cycle (ICAC)}{Universit\'e de Lille, France}{}{}{}

\section{Computational skills}

\cvitem{Programming languages}{Fortran, C and C++ (HPC optimization and parallelization with OpenMP, MPI, ScaLAPACK, CUDA and HIP), Python (SciPy, NumPy, Matplotlib, scikit-learn), Julia, Wolfram Language, Bash, Lisp, \LaTeX, HTML/CSS}
\cvitem{Software}{\href{https://fhi-aims.org/who-we-are}{\textcolor{unazul}{FHI-aims}} and \href{https://nomad-coe.github.io/greenX/}{\textcolor{unazul}{GreenX}} (Developed low-scaling massively parallel GW algorithms), \href{https://www.pci.uni-heidelberg.de/tc/usr/mctdh/ref.html}{\textcolor{unazul}{MCTDH}} (Developed the SOP-FBR and SRPTucker packages), \href{https://trex-coe.eu/trex-quantum-chemistry-codes/champ}{\textcolor{unazul}{CHAMP}} (Developed massively parallel QMC algorithms), \href{https://rxnkin.usc.es/index.php/AutoMeKin}{\textcolor{unazul}{AutoMeKin}}, MOPAC, Quantum package, MOLPRO, MOLCAS, Gaussian, Wolfram Mathematica, \href{https://panadestein.github.io/emacsd/}{\textcolor{unazul}{Emacs}}, VS code, Git, Inkscape, LibreOffice}
\cvitem{Numerical methods}{Tensor decomposition, Optimization Algorithms, Quantum Monte Carlo, low-scaling GW algorithms}
\cvitem{OS}{(\href{https://github.com/Panadestein/nixos-config.git}{\textcolor{unazul}{NixOS}}) Linux, UNIX}

\section{Languages (CEFR)}

\cvitemwithcomment{Spanish}{Native speaker}{}
\cvitemwithcomment{English}{Proficient user (C1)}{}
\cvitemwithcomment{French}{Proficient user (C1)}{}
\cvitemwithcomment{German}{Proficient user (Goethe-Zertifikat B1)}{}

%----------------------------------------------------------------------------------------
%	Cover letter (uncomment on demand)
%----------------------------------------------------------------------------------------

\begin{comment}
\clearpage
\thispagestyle{empty}

\recipient{Person or organization}{Address}
\date{\today}
\opening{Dear Sir or Madam,}
\closing{Sincerely yours,}

\makelettertitle{}
\justify{}

Letter's body here

\makeletterclosing{}
\end{comment}

%----------------------------------------------------------------------------------------
%	Endorsements (uncomment in demand)
% ----------------------------------------------------------------------------------------

\begin{comment}

\clearpage

\thispagestyle{empty}
\begin{center}
  \LARGE \textbf{References}
\end{center}
\vspace{3cm}

\begin{itemize}
  \item \textbf{Reference 1}\\
        \vspace{-.5cm}
        \begin{changemargin}{}{}
          Address \\
          \textbf{Phone:} \\
          \textbf{Email:}
        \end{changemargin}

        \vspace{2cm}

  \item \textbf{Reference 2}\\
        \vspace{-.5cm}
        \begin{changemargin}{}{}
          Address \\
          \textbf{Phone:} \\
          \textbf{Email:}
        \end{changemargin}
\end{itemize}

\end{comment}

%----------------------------------------------------------------------------------------
%	Abstract of work (uncomment on demand)
% ----------------------------------------------------------------------------------------

\begin{comment}
\clearpage
\thispagestyle{empty}

\begin{center}
  \textbf{Abstract of the Doctoral Dissertation}
\end{center}

My PhD aims at simulating full quantum mechanically (nuclei and electrons) the processes of adsorption and photoreactivity
of NO\(_2\) adsorbed on soot particles (modeled as large Polycyclic Aromatic Hydrocarbons, PAHs) in atmospheric conditions.
A detailed description of these processes is necessary to understand the differential day-nighttime behavior of
the production of HONO~\cite{guan2017identification,monge2010light}, which is a precursor of the hydroxyl radical
(OH)~\cite{holloway2015atmospheric}. In particular, the specific mechanism of the soot-mediated interconversion
between NO\(_2\) and HONO is to date not fully understood. Due to  its particular relevance in this context,
we have chosen the Pyrene-NO\(_2\) as our model system~\cite{guan2017identification}.\\

The first stage in this study has consisted of the determination of the stable configurations (transition states and minima)
of the Pyrene-NO\(_2\) system. To this end, we have used the recently developed van der Waals Transition State Search
using Chemical Dynamics Simulations (vdW-TSSCDS) method~\cite{kopec2019vdw}, the generalization of the TSSCDS algorithm
developed in our group. In this way, the present work represents the first application of vdW-TSSCDS to a large system
(81D)~\cite{panades2021}. Starting from a set of judiciously chosen input geometries, the aforementioned method permits
the characterization of the topography of an intermolecular Potential Energy Surface (PES), or in other words the determination
of the most stable conformations of the system, in a fully automated and efficient manner. \\

The gathered topographical information has been used to obtain a global description (fit) of the interaction potential,
necessary for the dynamical elucidation of the intermolecular interaction (physisorption), spectroscopic properties,
and reactivity of the adsorbed species. To achieve this last goal, we have developed two different methodologies together
with the corresponding software packages. The first one of them is the Specific Reaction Parameter Multigrid POTFIT (SRP-MGPF)
algorithm, which is implemented in the SRPTucker package~\cite{panades2019specific}. This method computes chemically accurate
(intermolecular) PESs through the reparametrization of semiempirical methods, which are subsequently tensor decomposed
into Tucker form using MGPF~\cite{pel13:014108}. This software has been successfully interfaced with the Heidelberg version
of the Multi-Configuration Time-Dependent Hartree (MCTDH) package~\cite{wor00:v81}. The second method allows for obtaining the
PES directly in the mathematical form required by MCTDH, thence its name Sum-Of-Products Finite-Basis-Representation
(SOP-FBR)~\cite{panades2020sop}. SOP-FBR constitutes an alternative approach to Neural Networks (NN)-fitting methods.
The idea behind it is simple: from the basis of a low-rank Tucker expansion on the grid, we replace the grid-based basis
functions by an expansion in terms of orthogonal polynomials. As in the previous method, smooth integration with MCTDH has
been ensured. Both methods have been successfully benchmarked with a number of reference problems, namely: the Hénon-Heiles
Hamiltonian, a global H\(_2\)O PES, and the HONO isomerization PES (6D). \\

With the aid of all the above-mentioned methods, we have tackled the computation of the global PES of the Pyrene-NO\(_2\) system.
Suitable coordinate transformation routines have been developed to map the Cartesian coordinates to internal coordinates.
In the physisorption domain, the evidence collected with vdW-TSSCDS has suggested that the geometry of the NO\(_2\) molecule
is almost not perturbed in the stationary points with respect to the isolated molecule. This fact has enabled its treatment
in a rigid monomer fashion (6D). The PESs will be used to obtain the electronic ground state (GS) and corresponding
Zero-Point Energy (ZPE) of the system  with MCTDH\@. The ZPE can offer an accurate estimate of the adsorption energy
of the NO\(_2\) molecule over the Pyrene. Additionally, the electronic absorption spectrum of the system will be obtained
by computing the sum (weighted by the GS distribution) of the individual vertical excitations of each stationary point.\\

\bibliographystyle{unsrt}
\bibliography{refs}
\end{comment}

\end{document}
