\documentclass[12pt]{article}

% ----------------------------------------------------------------------------------------
%	Packages
% ----------------------------------------------------------------------------------------
\usepackage[english]{babel}
\usepackage{cmbright}
\usepackage{enumitem}
\usepackage{fancyhdr}
\usepackage{fontawesome5}
\usepackage{geometry}
\usepackage{hyperref}
\usepackage[sf]{libertine}
\usepackage{microtype}
\usepackage{paracol}
\usepackage{supertabular}
\usepackage{titlesec}
\hypersetup{colorlinks, urlcolor=black, linkcolor=black}
\usepackage{xcolor}
\definecolor{unazul}{HTML}{337ab7}

% Geometry
\geometry{hmargin=1.75cm, vmargin=2.0cm}
\columnratio{0.65, 0.35}
\setlength\columnsep{0.05\textwidth}
\setlength\parindent{0pt}
\setlength{\smallskipamount}{8pt plus 3pt minus 3pt}
\setlength{\medskipamount}{16pt plus 6pt minus 6pt}
\setlength{\bigskipamount}{24pt plus 8pt minus 8pt}

% Style
\pagestyle{empty}
\titleformat{\section}{\color{unazul}\scshape\LARGE\raggedright}{}{0em}{}[\titlerule]
\titlespacing{\section}{0pt}{\bigskipamount}{\smallskipamount}
\newcommand{\heading}[2]{\centering{\sffamily\Huge #1}\\\smallskip{\parbox{\linewidth}{\large{#2}}}}
\newcommand{\entry}[4]{{{\textbf{#1}}} \hfill #3 \\ #2 \hfill #4}
\newcommand{\tableentry}[3]{\textsc{#1} & #2\expandafter\ifstrequal\expandafter{#3}{}{\\}{\\[6pt]}}

  \begin{document}

  \begin{paracol}{2}

    % ----------------------------------------------------------------------------------------
    %	Contact information
    % ----------------------------------------------------------------------------------------

    \heading{Ramón L. Panadés Barrueta}{Physicist with robust background in software engineering.
      Proven expertise in numerical analysis and quantum computing. Developer of HPC,
      massively parallel applications for both CPUs and accelerators.}

    \switchcolumn{}

    \begin{supertabular}{ll}
      \\ % Dirty hack
      \textcolor{unazul}{\footnotesize\faEnvelope} & \href{mailto:rpana92@gmail.com}{rpana92@gmail.com} \\
      \textcolor{unazul}{\footnotesize\faGlobe} &
      \href{https://panadestein.github.io}{panadestein.github.io} \\
      \textcolor{unazul}{\footnotesize\faGithub} &
      \href{https://github.com/Panadestein}{github.com/Panadestein} \\
      \textcolor{unazul}{\footnotesize\faLinkedin} &
      \href{https://www.linkedin.com/in/rpanades}{linkedin.com/in/rpanades} \\
    \end{supertabular}

    \switchcolumn*

    % ----------------------------------------------------------------------------------------
    % CV content
    % ----------------------------------------------------------------------------------------
    \section{experience}

    \entry{Algorithmiq}{Software engineer}{Finland (remote)}{2024--present}
    \begin{itemize}[noitemsep,leftmargin=3.5mm,rightmargin=0mm,topsep=6pt]
    \item Implement high-performance, hybrid quantum-classical algorithms utilizing tensor-network methods
    \item Disseminate coding best practices and extensible software design, 
          contribute to knowledge enhancement through code reviews and pair programming
    \end{itemize}

    \entry{TU Dresden}{Postdoctoral researcher}{Dresden, Germany}{2021--2023}
    \begin{itemize}[noitemsep,leftmargin=3.5mm,rightmargin=0mm,topsep=6pt]
    \item Implemented new algorithms in FHI-aims/GreenX, achieving increased scalability in PDE solvers and GPU porting via CUDA and HIP
    \item Led teams in NVIDIA and AMD hackathons, focusing on porting tensor contraction algorithms to accelerators
    \end{itemize}

    \entry{University of Twente}{Postdoctoral researcher}{Enschede, Netherlands}{2020--2021}
    \begin{itemize}[noitemsep,leftmargin=3.5mm,rightmargin=0mm,topsep=6pt]
    \item Created stochastic solvers for PDEs, optimized for exascale HPC architectures, achieving 26\% speedup
    \item Enhanced QMCkl library with analytic derivatives of wave functions; led technical training in computational methods
    \end{itemize}

    %\entry{University of Heidelberg}{Graduate Research Assistant}{Heidelberg, Germany}{Oct--Dec 2019}
    %\begin{itemize}[noitemsep,leftmargin=3.5mm,rightmargin=0mm,topsep=6pt]
    %\item Developed a Python tool for efficient high-dimensional analysis, interfacing with SCF solvers written in Fortran
    %\end{itemize}

    \switchcolumn{}

    \section{skills}
    \begin{supertabular}{rl}
      \tableentry{\footnotesize\faCode}{Fortran \textperiodcentered{} C
        \textperiodcentered{} C++ \textperiodcentered{} Python}{}
      \tableentry{}{Julia \textperiodcentered{} Lisp \textperiodcentered{} BQN \textperiodcentered{} Bash/Fish/Nu}{}
      \tableentry{}{LaTeX \textperiodcentered{} HTML/CSS}{}
      \tableentry{\footnotesize\faCogs}{Qiskit \textperiodcentered{} SciPy \textperiodcentered{} Numpy \textperiodcentered{} CuPy}{}
      \tableentry{}{MPI \textperiodcentered{} OpenMP \textperiodcentered{} ScaLAPACK}{}
      \tableentry{}{CUDA \textperiodcentered{} HIP}{}
      \tableentry{\footnotesize\faLaptop}{git \textperiodcentered{} Nix/Docker \textperiodcentered{}  Emacs \textperiodcentered{} VS Code}{}
      \tableentry{}{GNU/Linux ecosystem \textperiodcentered{} SLURM}{}

      \tableentry{\footnotesize\faLanguage}{English \textperiodcentered{} advanced}{}
      \tableentry{}{French \textperiodcentered{} advanced}{}
      \tableentry{}{German \textperiodcentered{} Goethe-Zertifikat B1}{}
      \tableentry{}{Spanish \textperiodcentered{} native}{}
    \end{supertabular}

    \section{awards}
    \begin{supertabular}{rl}
      \tableentry{2018--2022}{\textbf{4 Poster Prizes}}{}
      \tableentry{}{Psi-k}{}
      \tableentry{}{HDQD}{}
      \tableentry{}{IMAMPC (twice)}{spaceafter}
      \tableentry{2010--2015}{\textbf{Science Olympiads}}{}
      \tableentry{}{ICPC Caribbean Finals}{}
      \tableentry{}{$42^{nd}$ IChO, Japan}{spaceafter}
    \end{supertabular}
    
    \switchcolumn*

    \vspace{-1.0cm}
    \section{education}

    \entry{University of Lille}{PhD in Physics}{Lille, France}{2017--2020}
    \begin{itemize}[noitemsep,leftmargin=3.5mm,rightmargin=0mm,topsep=6pt]
    %\item Engineered fully analytical, heuristic optimization algorithms in Python and Fortran, 
    %  offering an efficient alternative to expensive methods like HOOI and HOSVD
    \item Specialized in tensor decomposition via CP and Tucker forms, leveraging advanced
      linear algebra techniques for enhanced accuracy
    \end{itemize}

    \entry{University of Lille}{MSc in Physics}{Lille, France}{2016--2017}

    \entry{University of Havana}{BSc in Radiochemistry}{Havana, Cuba}{2011--2016}

    \switchcolumn{}

    \vspace{-0.3cm}
    \section{other interests}

    \begin{supertabular}{rl}
      \tableentry{(\(\iota\))}{Array languages}{}
      \tableentry{}{APL and BQN}{spaceafter}
      \tableentry{(\(\lambda\))}{Functional programming}{}
      \tableentry{}{Common Lisp and NixOS}{spaceafter}
      \tableentry{FOSS}{Emacs contributor}{}
    \end{supertabular}

  \end{paracol}

\end{document}